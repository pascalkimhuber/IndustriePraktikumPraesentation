\documentclass[a4paper]{scrartcl}

\usepackage[ngerman]{babel} % German output
\usepackage[utf8]{inputenc} % Input encoding
\usepackage{todonotes} % Notes





\begin{document}

\begin{titlepage}
  \begin{center}
    \sf
    Universität Bonn, Mathematisch-Naturwissenschaftliche Fakultät, Master of Science, Mathematics \\
    \vspace{1cm}
    \today \\
    \vspace{3cm}
    {\LARGE Praktikumsbericht von} \\
    \vspace{1cm}
    Pascal Huber (2219408), 8. Fachsemester, pascal.huber@aol.com \\
    \vspace{2cm}
    {\LARGE über ein Praktikum beim Fraunhofer-Institut für Eingebettete Systeme und Kommunikationstechnik ESK} \\
    \vspace{1cm}
    vom 15.04.2013 bis zum 27.09.2013 \\
  \end{center}
\end{titlepage}


\tableofcontents
\newpage

\section{Einleitung}
\label{sec:einleitung}



\section{Beschreibung der Praktikumsstelle}
\label{sec:beschr-der-prakt}

Das Fraunhofer-Institut für Eingebettete Systeme und Kommunikationstechnik ESK (kurz ESK) ist Teil der Fraunhofer-Gesellschaft zur Förderung der angewandten Forschung e.V. 

\todo[inline]{Kurze Einleitung und Überblick, was hier noch so kommt.}

\subsection{Die Fraunhofer Gesellschaft zur Förderung der angewandten Forschung e.V.}
\label{sec:die-fraunh-gesellsch} %

Die Fraunhofer Gesellschaft sieht ihre zentrale Aufgabe in der Förderung von anwendungsorientierter Forschung zum unmittelbaren Nutzen für die Wirtschaft und zum Vorteil für die Gesellschaft. Zu diesem Zweck betreibt sie Vorlaufs- und Auftragsforschung in den Forschungsfeldern Gesundheit, Sicherheit, Kommunikation, Mobilität, Energie und Umwelt. 

Die Fraunhofer Gesellschaft ist mit ungefähr 22.000 Mitarbeitern an 40 Standorten die größte Organisation für anwendungsorientierte Forschung in Europa und stellt somit einen Eckpfeiler der deutschen Forschungslandschaft dar. 
Das gesamte Finanzvolumen der Fraunhofer Gesellschaft betrug im Jahr 2012 \todo{Fußnote mit Quelle angeben} ungefähr 1.9 Mrd. Euro, wobei der Großteil (etwa 70\%) der Erträge durch Aufträge aus der Wirtschaft und von öffentlichen Forschungsprojekten erwirtschaftet wurde. Der restliche Anteil des Forschungsvolumens erhält die Fraunhofer Gesellschaft als Grundfinanzierung von Bund und Ländern. 

Intern ist die Fraunhofer Gesellschaft in 66 eigenständige Institute und einige weitere Einrichtungen unterteilt die von einer Zentrale mit Sitz in München koordiniert werden. Die einzelnen Einrichtungen arbeiten weitestgehend unabhängig und agieren selbständig auf dem Markt, was auch die Finanzierung durch Auftragsforschung betrifft. \todo{Verbünde erwähnen?}


\subsection{Das Fraunhofer-Institut für Eingebettete System und Kommunikationstechnik ESK und das Geschäftsfeld Automotive} \todo{Überschrift zu lang?}
\label{sec:das-fraunh-inst}

Das ESK ging 1999 zunächst nur als Fraunhofer Einrichtung aus der Abteilung Systemtechnik/Telekommunikation des Fraunhofer Instituts für Festkörpertechnologie und wurde im Juli 2013 endgültig in ein dauerhaftes Fraunhofer Institut überführt. Das Institut hat seinen Sitz in unmittelbarer Nähe zur Fraunhofer Zentrale in München und ist mit ungefähr 80 \todo{Quelle angeben und hinzufügen, dass die meisten Mitarbeiter Informatiker und Elektrotechniker sind} ständigen Mitarbeitern eines der kleinsten Institute der Fraunhofer Gesellschaft. \\
Das Forschungsfeld des ESK sind Verfahren und Methoden der Informations- und Kommunikationstechnik (kurz IKT) mit Schwerpunkten auf der Adaptivität (anpassende Systeme) von verteilten und vernetzten Systemen und Software-Methodik. \todo{Wie weit soll ich hier ins Detail gehen?} Dies umfasst beispielsweise die Konzeption und Analyse neuer Ansätze in der Softwareentwicklung für vernetzte Systeme (Parallelisierung und Multicore-Software), die Entwicklung und Verbesserung von zuverlässigen und effizienten Funknetzen in industriellen Anlagen oder im Fahrzeugbereich (M2M, Car-to-X) \todo{Hier muss noch eine Erklärung hin oder soll ich es gleich ganz weglassen?} oder die Optimierung von Software für adaptive verteilte Systeme, d.h. Systeme, die sich selbständig an Änderungen in der Umgebung oder im Ressourcenbedarf anpassen (z. Bsp. Steuergeräte im Automobilbereich). 
\\\\
Das ESK gliedert sich in die drei Geschäftsfelder Automotive (Fahrzeugkommunikation), Industrial Communication (Software- und Kommunikationslösungen für industrielle Anwendungen) und Telecommunication (u.a. Verbesserung von Leitungsnetzen). Meine Praktikumsstelle wurde im Rahmen des Geschäftsfeldes Automotive ausgeschrieben, weshalb ich diesen und insbesondere den Leistungsbereich \emph{Automotive Connectivity} \todo{Hervorheben? Schreiben, dass Automotive größtes Geschäftsfeld ist} näher beschreiben möchte. \\
Um die Sicherheit und die Effizienz im Straßenverkehr zu steigern, ist eine immer stärkere Vernetzung von Fahrzeugen sowohl untereinander als auch mit der Umgebung notwendig. 
Das Geschäftsfeld Automotive arbeitet deshalb an Hardware- und Softwarelösungen, die die Zuverlässigkeit und die Dienstqualität dieser Vernetzung gewährleisten soll und unterstützt somit Fahrzeughersteller und Dienstleister bei der Steuergeräte-Entwicklung und der Fahrzeug-Umwelt-Vernetzung (sog. Car-to-X Systeme). \\
In diesem Rahmen entwickelt der Leistungsbereich Automotive Connectivity Software-Frameworks, die die aus der Vernetzung entstehenden Daten verarbeiten und somit zum Beispiel die Integration von kooperativen Fahrerassistenzsystemen (z.Bsp. kooperativer Tempomat) ermöglichen. \todo{Zu viele Details? Ist das verständliche?}

\subsection{Das Bewerbungsverfahren}
\label{sec:das-bewerb}

Nachdem ich mich zum Ende des Wintersemesters 2012/13 entschlossen hatte, im darauf folgenden Sommersemester ein mehrmonatiges Industriepraktikum zu absolvieren, machte ich mich zu Beginn der Semesterferien auf die Suche nach einem geeigneten Praktikumsplatz im Großraum München. \todo{Soll ich das lieber weglassen?} Als Hauptinfomationsquelle diente mir das Internet. Dabei fand ich unter anderem auch die Stellenausschreibung für mein späteres Praktikum auf dem Karriereportal der Fraunhofer-Gesellschaft, für das ich mich online bewerben konnte. Nur wenige Tage nach der Bewerbung wurde ich telefonisch zu einem Bewerbungsgespräch eingeladen, in dem mich mein späterer Betreuer zu meiner Motivation und meinem Vorwissen befragte und anschließend den genauen Inhalt des Praktikums darlegte. Das Gespräch hatte weniger den Charakter eines Bewerbungsgesprächs, sondern war vielmehr eine nette Unterhaltung, an dessen Ende mir der Praktikumsplatz bereits zugesichert wurde. \todo{Fahrtkosten erwähnen?} Wenige Tage nach diesem Gespräch wurde der genaue zeitliche Rahmen des Praktikums festgelegt und ich erhielt meinen Arbeitsvertrag. \todo{Soll ich noch hinzufügen welche Erfahrungen ich aus der Bewerbung mitnehme?}


\subsection{Rahmenbedingungen des Praktikums}
\label{sec:rahm-des-prakt}

Meine Praktikumsstelle war als Vollzeitpraktikum ausgelegt, sodass ich 39 Stunden in der Woche im ESK arbeiten sollte. Es gab zwar keine festen Arbeitszeiten, allerdings mussten alle Mitarbeiter in bestimmten Kernzeiten anwesend sein. \\
Während meines Praktikums hatte ich einen festen Betreuer, der am selben Projekt arbeitete und den ich auch schon beim Bewerbungsgespräch kennen gelernt hatte. Einen festen Besprechungstermin gab es zwar nicht, allerdings konnte ich meinen Betreuer bei Fragen problemlos persönlich oder auch per E-Mail um Rat fragen. \todo{Mehr über die Betreuung schreiben?} \\
Da mein Praktikum über einen Zeitraum von mehr als fünf Monaten ging, erhielt ich einen eigenen Arbeitsplatz in einem der Praktikantenräume des ESK. Neben mir arbeiteten ungefähr 20 weitere Studenten als Hilfskräfte oder schrieben ihre Abschlussarbeiten im Institut, sodass die Praktikantenräume immer sehr gut gefüllt waren. \todo{Doofe Formulierung... der ganze Abschnitt gefällt mir nicht} 
\todo{Auch über Urlaubstage und Aufwandsentschädigung schreiben? (Habe ich im Internet gelesen...)}



\section{Quellen}
\label{sec:quellen}

- Homepage der Fraunhofer Gesellschaft: www.fraunhofer.de


\end{document}


