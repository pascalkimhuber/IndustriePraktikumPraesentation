\documentclass[a4paper, twoside]{scrartcl}

\usepackage[ngerman]{babel} % German output
\usepackage[utf8]{inputenc} % Input encoding
\usepackage{todonotes} % Notes





\begin{document}

\begin{titlepage}
  \begin{center}
    \sf
    Universität Bonn, Mathematisch-Naturwissenschaftliche Fakultät, Master of Science, Mathematics \\
    \vspace{1cm}
    \today \\
    \vspace{3cm}
    {\LARGE Praktikumsbericht von} \\
    \vspace{1cm}
    Pascal Huber (2219408), 8. Fachsemester, pascal.huber@aol.com \\
    \vspace{2cm}
    {\LARGE über ein Praktikum beim Fraunhofer-Institut für Eingebettete Systeme und Kommunikationstechnik ESK} \\
    \vspace{1cm}
    vom 15.04.2013 bis zum 27.09.2013 \\
  \end{center}
\end{titlepage}


\tableofcontents
\newpage

\section{Einleitung}
\label{sec:einleitung}



\section{Beschreibung der Praktikumsstelle}
\label{sec:beschr-der-prakt}

Das Fraunhofer-Institut für Eingebettete Systeme und Kommunikationstechnik ESK (kurz ESK) ist Teil der Fraunhofer-Gesellschaft zur Förderung der angewandten Forschung e.V. 

\todo[inline]{Kurze Einleitung und Überblick, was hier noch so kommt.}

\subsection{Die Fraunhofer Gesellschaft zur Förderung der angewandten Forschung e.V.}
\label{sec:die-fraunh-gesellsch}

Die Fraunhofer Gesellschaft sieht ihre zentrale Aufgabe in der Förderung von anwendungsorientierter Forschung zum unmittelbaren Nutzen für die Wirtschaft und zum Vorteil für die Gesellschaft. Zu diesem Zweck betreibt die Fraunhofer Gesellschaft Vorlaufs- und Auftragsforschung in den Forschungsfeldern Gesundheit, Sicherheit, Kommunikation, Mobilität, Energie und Umwelt. 

Die Fraunhofer Gesellschaft ist mit ungefähr 22 000 Mitarbeitern an 40 Standorten die größte Organisation für anwendungsorientierte Forschung in Europa und stellt somit einen Eckpfeiler der deutschen Forschungslandschaft dar. 




\todo{1. Ziel der Fraunhofer Gesellschaft, Struktur der Fraunhofer Gesellschaft}


\section{Quellen}
\label{sec:quellen}

- Homepage der Fraunhofer Gesellschaft: www.fraunhofer.de


\end{document}


