\section{Mathematischer Bezug}
% ** Mathematik im Praktikum

% + *mathematisches Fachwissen ist von Vorteil, aber nicht zwingend notwendig*
%   - *Grundkenntnisse in der Graphentheorie (Dijkstra etc.)*
%   - *Kenntnisse in Geometrie für den Umgang mit dem WGS84 Koordinatensystem*
%   - *Für kompliziertere Mapmatching-Algorithmen: Stochastik (Kalman Filter), Logik (Fuzzy Logic)*
% + *Wichtiger ist "mathematische" Denkweise* /Anmerkung: strukturiertes, logisches Denken/
%   - *organisierte Arbeitsweise*
%   - *unüberschaubare Problemstellungen auf Kernprobleme reduzieren (zum Beispiel: Probleme in Teilprobleme unterteilen)* /Anmerkung: Abstraktionfähigkeit, zum Beispiel geometrischer Mapmatcher/
%   - *Anwenden von abstrakten Theorien auf Probleme der realen Welt (zum Beispiel: Berücksichtigung von Abbiegeverboten im Dijkstra-Algorithmus)* /Aufgabenlösen im Studium/
% + *Schnelle und selbständige Einarbeitung in neue Themenbereiche* /Im Mathematikstudium durch Seminare/
%   - *Verständnis und Bewertung von neuen Algorithmen*
%   - *Zurechtfinden in neuen Datenstrukturen* /Anmerkung: großes Software repository, Software Bibliotheken/
% + *Hohe Frustrationstoleranz*
% + *Gute Fähigkeiten am Computer* /Anmerkung: ist nicht wirklich Mathematik-Typisch, aber Programmieraufgaben sind gut.../

%------------------------------------------------------------------------
\mode<presentation>{
  \begin{frame}[t] \frametitle{Mathematischer Bezug im Praktikum}
    \vspace{0.5cm}
    \begin{itemize}
    \item mathematisches Fachwissen ist von Vorteil, aber nicht zwingend
      \begin{itemize}
      \item Grundkenntnisse der diskreten Mathematik (Graphentheorie, Dijkstra-Algorithmus)
      \item Geometrie f\"ur Distanz-Berechnung auf gekr\"ummten Fl\"achen
      \item Stochastik (Kalman Filter), Logik (Fuzzy Logic) f\"ur kompliziertere Mapmatching-Algorithmen
      \end{itemize}
      \vspace{0.5cm}
    \item Wichtig: \emph{``mathematische''} Denkweise
      \begin{itemize}
      \item Reduktion auf Kernprobleme bei un\"uberschaubaren Problemstellungen (z.Bsp. Probleme in Teilprobleme unterteilen)
      \item Anwenden von abstrakten Theorien auf Probleme der realen Welt (z. Bsp. Ber\"ucksichtigung von Abbiegeverboten im Dijkstra-Algorithmus)
      \end{itemize}
    \end{itemize}
  \end{frame}  
}

%------------------------------------------------------------------------

\begin{frame}[t] \frametitle{Mathematischer Bezug im Praktikum}
  \vspace{1cm}
  \begin{itemize}
  \item Eingenst\"andiges und schnelles Einarbeiten in neue Themenbereiche
    \begin{itemize}
    \item Verst\"andnis und Bewertung von neuen Algorithmen
    \item Zurechtfinden in neue Datenstrukturen
    \end{itemize}
  \end{itemize}
\end{frame}

%------------------------------------------------------------------------
